
{\actuality} \epigraph{Все есть поток}{\textit{Гераклит}}
Морской лед является важной частью природной системы Земли, которая регулирует теплообмен между океаном и атмосферой~\cite{vaughan2013observations}. Результаты анализа данных ERA"~Interim (ERAI)~\cite{dee2011era} и Национального центра по снегу и льду (NSIDC, США)~\cite{cavalieri1996updated}, указывают на сокращение площади занимаемой морским льдом, начиная с конца XX века, особенно в Баренцевом и Карском морях, а также в районе моря Баффина "--- тренд оценивается в -0.24\% в год~\cite{park2015attribution}. Сокращение площади ледяного покрова наблюдается в течении последних 35 лет, согласно анализу данных спутниковых наблюдений \cite{simmonds2015comparing}. Однако отсутствие систематических данных наблюдений о толщине льда не позволяют сделать подобную оценку изменений этого параметра. Совместный анализ измерений выполненных с подводных лодок и данных спутниковой альтиметрии (IceSAT) указывает на сокращение толщины многолетних льдов в период с 1980 по 2008 г. на 1.75 м в зимний период и на 1.65 м летом \cite{rothrock2008decline,kwok2009thinning}. Происходящие климатические изменения по мнению некоторых ученых могут привести к условиям безледной обстановке в летний период в Арктике уже в XXI веке  \cite{stroeve2007arctic,wang2009sea,massonnet2012constraining,Laliber_2016,Jahn_2016}. Другие исследования указывают на циклический характер этих изменений, анализ которых по мнению авторов указывает на предстоящий период похолодания \cite{Frolov2010,Frolov2007_1,Frolov2007_2}.

Дрейф льда в физическом смысле предаставляющий собой направленный перенос массы "--- другой важный параметр морского льда, который является одним из факторов определяющих ледовый баланс, регулируя образование пространств открытой воды "--- полыней, разводий и трещин, которые обуславливают процессы ледообразования, а также существенно влияют на теплообмен между атмосферой и океаном, солевой баланс воды \cite{jordan1999heat,Makshtas1977}. Надежные данные о полях дрейфа являются необходимым условием не только диагностической оценки происходящих режимных изменений, но и для верификации численных моделей, которые являются основой прогнозов будущего состояния льда. Анализ результатов расчетов с использованием современных климатических  моделей показывает недооценку значений площади распространения и толщины морского льда в Арктике \cite{stroeve2007arctic}, что может быть вызвано некорректным воспроизведением кинематики льда (дрейфа и деформации)  \cite{rampal2011ipcc}. Также учет динамики льда имеет большое значение при планировании и проведении морских операций и эксплуатации инженерных сооружений. Т.о. дрейф льда является одним из ключевых природных процессов в Арктике, представляющих большой интерес как для океанографических исследований, в том числе численного моделирования морского льда, так и для обеспечения безопасного плавания во льдах.

Современные спутниковые системы активной радиолокационной съемки с синтезированием апертуры (SAR)  \cite{sar} позволяют проводить наблюдения за морским льдом независимо от освещенности подстилающей поверхности и наличия облачного покрова. Вместе с тем сложность физической природы формирования отраженного радиолокационного сигнала требует особых усилий и учета его особенностей при разработке методов его интерпретации "--- перевода в реальную природную физическую величину. Настоящая работа посвящена совершенствованию методов мониторинга (под которым понимается \textit{"--- наблюдение за какими-нибудь процессами для оценки их состояния и прогнозов развития}) дрейфа льда на основе спутниковых радиолокационных данных.

Широкое использование и доступность SAR"~изображений, с начала 1990-х годов, а также запуск новых спутников имеющих на борту инструменты радиолокационной съемки, требуют развития методов восстановления информации о морском льду, в том числе дрейфа льда, а также ее анализа и интеграции в морские информационные системы. Современные подходы к восстановлению полей дрефа льда можно разделить на три группы: (1) дифференциальные методы, (2) блочные, (3) прослеживание локальных особенностей изображений. Первая группа основывается на методе оптического потока для каждого пикселя изображения, как например в \cite{sun1996automatic}. Вторая группа представляет собой корреляционный анализ <<блоков>> изображений, одной из классических работой в этом направлении яявляется \cite{fily1986extracting}. Третья группа оперирует так называемыми <<особенностями>> детектированных на изображениях, в качестве которых могут выступать описания контуров объектов, формализованные особые точки и~т.\:п. \cite{daida1990object,mcconnell1991psi} 
Большую важность, по мнению автора работы, представляет улучшение качества методов восстановления полей дрейфа льда, на основе автоматического анализа последовательных SAR"~изображений. Широкая доступность этих данных для исследовательских целей, высокое пространственное разрешение (десятки метров) и возможность всепогодной съемки делает их использование привлекательным для получения подробных данных о динамике льда.



% {\progress} 
% Этот раздел должен быть отдельным структурным элементом по
% ГОСТ, но он, как правило, включается в описание актуальности
% темы. Нужен он отдельным структурынм элемементом или нет ---
% смотрите другие диссертации вашего совета, скорее всего не нужен.

{\aim} данной работы является развитие и совершенствование методов автоматического восстановления полей дрейфа льда, их анализа, а также создание технологических решений оперативного ледового мониторинга

Для~достижения поставленной цели необходимо было решить следующие {\tasks}:
\begin{enumerate}
  \item Исследовать современные методы прослеживания особых точек на изображениях с целью их доработки и адаптации к SAR"~данным
  \item Реализовать улучшенный алгоритм дрейфа льда на основе результатов проведенного исследования
  \item Реализовать физически обоснованный метод анализа временных рядов векторных процессов для полей дрейфа льда и продемонстрировать возможности его применения на примере обработки спутниковых данных
  \item Разработать оперативную подсистему ледового мониторинга на примере восстановления полей дрейфа льда и прогноза дрейфа айсбергов
\end{enumerate}


{\novelty}
\begin{enumerate}
  \item Было выполнено оригинальное исследование по сопоставлению эффективности современных методов из области Компьютерного Зрения для решения задачи восстановления полей дрейфа льда на основе обработки спутниковых SAR"~изображений, выявлен наиболее эффективный подход, учитывающий физическую природу формирования SAR"~изображений, выполнена оценка качества получаемых векторов дрейфа
  \item Разработан оригинальный метод восстановления полей дрейфа льда, превосходящий по плотности получаемых данных современные алгоритмы на основе фильтрации изображений с использованием нелинейной диффузии, за счет наиболее эффективного подавления шума, характерного для SAR"~изображений
  \item Впервые применён Векторно"--~Алгебраический метод для анализа полей дрейфа льда в Арктике, полученных на основе обработки спутниковых данных, с помощью которого получены и проанализированы ранее не доступные характеристики динамики морского льда
  \item Созданы технологические схемы оперативного расчёта динамических характеристик морского льда (дрейфа и деформации) и прогностических траекторий дрейфа айсбергов на основе обработки SAR"~данных, обеспечивающих повышенную производительность всей цепочки расчётов
\end{enumerate}

{\influence} выполненной работы заключается в возможности использования результатов расчётов с использованием разработанного алгоритма для получения таких важных динамических характеристик динамики льда как дрейф и деформации (зоны разрежений и сжатий) для задач по обеспечению безопасного плавания во льдах, а также верификации и ассимиляции в численные модели с целью улучшения качества прогноза состояния ледяного покрова. 
Разработанная технология оперативного мониторинга и прогноза дрейфа айсбергов на основе SAR"~данных успешно использовалась при проведении разведочного бурения в Карском море, в 2014 г.

{\methods} В ходе исследования применялись различные методы, обусловленные теми задачами, которые решались в ходе работы:
\noindent
\begin{itemize}
	\item метод фильтрации на основе применения нелинейной диффузии для обработки SAR"~изображений.
	\item Векторно"~алгебраический метод анализа временных рядов для векторных процессов.
Использовались стандартные и специально разработанные алгоритмы и программы.
\end{itemize}


{\defpositions}
\begin{enumerate}
  \item Установлено, что метод фильтрации изображений с использованием нелинейной диффузии, при создании его многомасштабного представления, позволяет получить лучшие результаты при решении задачи восстановления полей дрейфа льда на основе обработки последовательных SAR"~изображений, в сравнение с другими существующими современными методами, основанными на методах линейной фильтрации
  \item Показано, что разработанный оперативный алгоритм восстановления полей дрейфа льда по последовательным спутниковым SAR"~изображениям наиболее эффективен с точки зрения баланса между количеством получаемых корректных векторов дрейфа и временем выполнения расчётов
  \item Показано, что применение Векторно"~алгебраического метода анализа векторных процессов для характеристики полей дрейфа морского льда позволяет комплексно описать закономерности его изменчивости, что затруднительно с использованием распространённого компонентного метода, использование которого приводит к разрыву связи внутри единой характеристики
  \item Установлено, что продукция на основе оперативной обработки SAR"~данных, благодаря повышению информативности получаемых характеристик, может эффективно применяться для решения задач ледового мониторинга в рамках комплексных задач по обеспечению безопасного плавания на примере морей Западной Арктики
\end{enumerate}

{\reliability} полученных результатов обеспечивается их верификацией на основе подспутниковых и экспертных данных, а также тем что результаты полученные в данном исследовании находятся в соответствии с результатами, полученными другими авторами. 

{\probation}
Основные результаты работы докладывались на следующих российских и зарубежных конференциях, симпозиумах и~семинарах: международной конференции по геонаукам и дистанционному зондированию IGARSS по эгидой IEEE (Форт"~Уэрт, 2017); всероссийском симпозиуме <<Радиолокационное исследование природных сред>> (Санкт"~Петербург, 2017); II Всероссийской конференции молодых ученых <<Комплексные исследования Мирового Океана>> (Москва, 2017); на заседании Океанографической комиссии Русского Географического общества (Санкт"~Петербург, 2017); на XIV Всероссийской Открытой конференции <<Современные проблемы дистанционного зондирования Земли из космоса>> (Москва, 2016); на Арктическом конгрессе (Санкт"~Петербург, 2016); Симпозиуме <<Живая Планета>> под эгидой Европейского космического агентства (ESA) (Прага, 2016); XIII Всероссийской Открытой конференции <<Современные проблемы дистанционного зондирования Земли из космоса>> (Москва, 2015); IV Международной научно-практическая конференция <<Морские исследования и образование>> (<<MARESEDU-2015>>) (Москва, 2015); конференции Междисциплинарных полярных исследований на Шпицбергене (<<IPSiS>>) (Лонгиир, 2015); Международном симпозиуме <<Атмосферная радиация и динамика "--- МСАРД 2015>> (Петродворец, 2015); V Международной конференции <<Логистика в Арктике>> (Мурманск, 2015); XX Всероссийской Открытой конференции <<Современные проблемы дистанционного зондирования Земли из космоса>> (Москва, 2014); заседании Океанографической комиссии Русского географического общества (Санкт"~Петербург, 2014); симпозиуме <<Живая Планета>> под эгидой Европейского космического агентства (ESA) (Эдинбург, 2013); XV Гляциологическом симпозиуме <<Прошлое, настоящее и будущее криосферы Земли>> (Архангельск, 2012);

{\contribution} Автор является идеологом и главным создателем алгоритма восстановления полей дрейфа льда по последовательным SAR"~изображениям с использованием уравнений нелинейной диффузии при построении многомасштабного пространства. Исследование современных методов основанных на прослеживании локальных особенностей на изображениях, с учётом специфики природы шума характерного для радиолокационных изображений, позволили создать эффективное технологическое решение задачи мониторинга дрейфа и деформации морского льда с использованием современных спутниковых средств. Автор принимал активное участие в создании технологии оперативного мониторинга дрейфа айсбергов в морях Западной Арктики, которая с 2014 года успешно применяется при обеспечении проведения морских операций.

%\publications\ Основные результаты по теме диссертации изложены в ХХ печатных изданиях~\cite{Sokolov,Gaidaenko,Lermontov,Management},
%Х из которых изданы в журналах, рекомендованных ВАК~\cite{Sokolov,Gaidaenko}, 
%ХХ --- в тезисах докладов~\cite{Lermontov,Management}.

\ifnumequal{\value{bibliosel}}{0}{% Встроенная реализация с загрузкой файла через движок bibtex8
    \publications\ Основные результаты по теме диссертации изложены в XX печатных изданиях, 
    X из которых изданы в журналах, рекомендованных ВАК, 
    X "--- в тезисах докладов.%
}{% Реализация пакетом biblatex через движок biber
%Сделана отдельная секция, чтобы не отображались в списке цитированных материалов
    \begin{refsection}[vak,papers,conf]% Подсчет и нумерация авторских работ. Засчитываются только те, которые были прописаны внутри \nocite{}.
        %Чтобы сменить порядок разделов в сгрупированном списке литературы необходимо перетасовать следующие три строчки, а также команды в разделе \newcommand*{\insertbiblioauthorgrouped} в файле biblio/biblatex.tex
        \printbibliography[heading=countauthorvak, env=countauthorvak, keyword=biblioauthorvak, section=1]%
        \printbibliography[heading=countauthorconf, env=countauthorconf, keyword=biblioauthorconf, section=1]%
        \printbibliography[heading=countauthornotvak, env=countauthornotvak, keyword=biblioauthornotvak, section=1]%
        \printbibliography[heading=countauthor, env=countauthor, keyword=biblioauthor, section=1]%
        \nocite{%Порядок перечисления в этом блоке определяет порядок вывода в списке публикаций автора
                Demchev_2017,Demchev_2016,%
                Volkov2016,Kulakov_2015,%
                Mironov2015,Volkov_2012,%
                Buixad_Farr__2014,Demchev2009,Kazakov2016,%
        }%
        \publications\ Основные результаты по теме диссертации изложены в \arabic{citeauthor} печатных изданиях, 
        \arabic{citeauthorvak} из которых изданы в журналах, рекомендованных ВАК, 
        \arabic{citeauthorconf} "--- в тезисах докладов.
    \end{refsection}
    \begin{refsection}[vak,papers,conf]%Блок, позволяющий отобрать из всех работ автора наиболее значимые, и только их вывести в автореферате, но считать в блоке выше общее число работ
        \printbibliography[heading=countauthorvak, env=countauthorvak, keyword=biblioauthorvak, section=2]%
        \printbibliography[heading=countauthornotvak, env=countauthornotvak, keyword=biblioauthornotvak, section=2]%
        \printbibliography[heading=countauthorconf, env=countauthorconf, keyword=biblioauthorconf, section=2]%
        \printbibliography[heading=countauthor, env=countauthor, keyword=biblioauthor, section=2]%
        \nocite{Demchev_2017,Demchev_2016,%
        	Volkov2016,Kulakov_2015,%
        	Mironov2015,Volkov_2012,%
        	Buixad_Farr__2014,Demchev2009}%vak
        \nocite{bib1}%notvak
        \nocite{Kazakov2016}%conf
    \end{refsection}
}
%При использовании пакета \verb!biblatex! для автоматического подсчёта
%количества публикаций автора по теме диссертации, необходимо
%их~здесь перечислить с использованием команды \verb!\nocite!.
