
{\actuality} \epigraph{Все есть поток}{\textit{Гераклит}}
Морской лед является важной частью природной системы Земли, которая регулирует теплообмен между океаном и атмосферой~\cite{vaughan2013observations}. Результаты анализа данных ERA"--~Interim (ERAI)~\cite{dee2011era} и Национального центра по снегу и льду (NSIDC, США)~\cite{cavalieri1996updated}, указывают на сокращение площади занимаемой морским льдом, начиная с конца XX века, особенно в Баренцеовом и Карском морях, а также в районе моря Баффина "--- тренд оценивается в -0.24\% в год~\cite{park2015attribution}. Сокращение площади ледяного покрова наблюдается в течении последних 35 лет, согласно анализу данных спутниковых наблюдений \cite{simmonds2015comparing}. Однако отсутствие систематических данных наблюдений о толщине льда не позволяют сделать подобную оценку изменений этого параметра. Совместный анализ измерений выполненных с подводных лодок и данных спутниковой альтиметрии (IceSAT) указывает на сокращение толщины многолетних льдов в период с 1980 по 2008 г. на 1.75 м в зимний период и на 1.65 м летом \cite{rothrock2008decline,kwok2009thinning}. Происходящие климатические изменения по мнению некоторых ученых могут привести к условиям безледной обстановке в летний период в Арктике уже в XXI веке  \cite{stroeve2007arctic,wang2009sea,massonnet2012constraining,Laliber_2016,Jahn_2016}. Другие исследования указывают на циклический характер этих изменений, анализ которых по мнению авторов указывает на предстоящий период похолодания \cite{Frolov2010,Frolov2007_1,Frolov2007_2}.

Дрейф льда в физическом смысле предаставляющий собой направленный перенос массы "--- другой важный параметр морского льда, который является одним из факторов определяющих ледовый баланс, регулируя образование пространств открытой воды "--- полыней, разводий и трещин, которые обуславливают процессы ледообразования, а также существенно влияют на теплообмен между атмосферой и океаном, солевой баланс воды \cite{jordan1999heat,Makshtas1977}. Надежные данные о полях дрейфа являются необходимым условием не только диагностической оценки происходящих режимных изменений, но и для верификации численных моделей, которые являются основой прогнозов будущего состояния льда. Анализ результатов расчетов с использованием современных климатических  моделей показыает недооценку значений площади распространения и толщины морского льда в Арктике \cite{stroeve2007arctic}, что может быть вызвано некорректным воспроизведением кинематики льда (дрейфа и деоформации)  \cite{rampal2011ipcc}. Также учет динамики льда имеет большое значение при планировании и проведении морских операций и эксплуатации инженерных сооружений. Т.о. дрейф льда является одним из ключевых природных процессов в Арктике, представляющих большой инетерес как для океанографических исследований, в том числе численного моделирования морского льда, так и для обеспечения безопасного плавания во льдах.

Современные сутниковые системы активной радиолокационной съемки с синтезированием аппертуры (SAR)  \cite{sar} повзволяют проводить наблюдения за морским льдом независимо от освещенности подстилающей поверхности и наличия облачного покрова. Вместе с тем сложность физической природы формирования отраженного радиолокационного сигнала требует особых усилий и учета его особенностей при разработке методов его интерпритаций "--- перевода в реальную природну физическую величину. Настоящая работа посвящена совершенствованию методов мониторинга (под которым понимается \textit{"--- наблюдение за какими-нибудь процессами для оценки их состояния и прогнозов развития}) дрейфа льда на основе спутниковых радилокационных данных.

Широкое использование и доступность SAR-изображений, с начала 1990-х годов, а также запуск новых спутников имеющих на борту истурменты радилокационной съемки, требуют развития методов восстановления информации о морском льду, в том числе дрейфа льда, а также ее анализа и интеграции в морские информационные системы. Современные подходы к восстановлению полей дрефа льда можно разделить на три группы: (1) дифференциальные методы, (2) блочные, (3) прослеживание локальных особенностей изображений. Перывая группа основывается на методе оптичексого потока для каждого пикселя изображения, как например в \cite{sun1996automatic}. Вторая группа представляет собой корреляционный анализ <<блоков>> изображений, одной из классических работой в этом направлении яявляется \cite{fily1987sea}. Третья группа оперирует так называемыми <<особенностями>> детектированных на изображениях, в качестве которых могут выступать описания контуров объектов, формализованные особые точки и т.п. \cite{daida1990object,mcconnell1991psi} 

 Большую важность, по мнению автора работы, представляет улучшение качества методов восстановления пол



% {\progress} 
% Этот раздел должен быть отдельным структурным элементом по
% ГОСТ, но он, как правило, включается в описание актуальности
% темы. Нужен он отдельным структурынм элемементом или нет ---
% смотрите другие диссертации вашего совета, скорее всего не нужен.

{\aim} данной работы является \ldots

Для~достижения поставленной цели необходимо было решить следующие {\tasks}:
\begin{enumerate}
  \item Исследовать, разработать, вычислить и~т.\:д. и~т.\:п.
  \item Исследовать, разработать, вычислить и~т.\:д. и~т.\:п.
  \item Исследовать, разработать, вычислить и~т.\:д. и~т.\:п.
  \item Исследовать, разработать, вычислить и~т.\:д. и~т.\:п.
\end{enumerate}


{\novelty}
\begin{enumerate}
  \item Впервые \ldots
  \item Впервые \ldots
  \item Было выполнено оригинальное исследование \ldots
\end{enumerate}

{\influence} \ldots

{\methods} \ldots

{\defpositions}
\begin{enumerate}
  \item Первое положение
  \item Второе положение
  \item Третье положение
  \item Четвертое положение
\end{enumerate}
В папке Documents можно ознакомиться в решением совета из Томского ГУ
в~файле \verb+Def_positions.pdf+, где обоснованно даются рекомендации
по~формулировкам защищаемых положений. 

{\reliability} полученных результатов обеспечивается \ldots \ Результаты находятся в соответствии с результатами, полученными другими авторами.


{\probation}
Основные результаты работы докладывались~на:
перечисление основных конференций, симпозиумов и~т.\:п.

{\contribution} Автор принимал активное участие \ldots

%\publications\ Основные результаты по теме диссертации изложены в ХХ печатных изданиях~\cite{Sokolov,Gaidaenko,Lermontov,Management},
%Х из которых изданы в журналах, рекомендованных ВАК~\cite{Sokolov,Gaidaenko}, 
%ХХ --- в тезисах докладов~\cite{Lermontov,Management}.

\ifnumequal{\value{bibliosel}}{0}{% Встроенная реализация с загрузкой файла через движок bibtex8
    \publications\ Основные результаты по теме диссертации изложены в XX печатных изданиях, 
    X из которых изданы в журналах, рекомендованных ВАК, 
    X "--- в тезисах докладов.%
}{% Реализация пакетом biblatex через движок biber
%Сделана отдельная секция, чтобы не отображались в списке цитированных материалов
    \begin{refsection}[vak,papers,conf]% Подсчет и нумерация авторских работ. Засчитываются только те, которые были прописаны внутри \nocite{}.
        %Чтобы сменить порядок разделов в сгрупированном списке литературы необходимо перетасовать следующие три строчки, а также команды в разделе \newcommand*{\insertbiblioauthorgrouped} в файле biblio/biblatex.tex
        \printbibliography[heading=countauthorvak, env=countauthorvak, keyword=biblioauthorvak, section=1]%
        \printbibliography[heading=countauthorconf, env=countauthorconf, keyword=biblioauthorconf, section=1]%
        \printbibliography[heading=countauthornotvak, env=countauthornotvak, keyword=biblioauthornotvak, section=1]%
        \printbibliography[heading=countauthor, env=countauthor, keyword=biblioauthor, section=1]%
        \nocite{%Порядок перечисления в этом блоке определяет порядок вывода в списке публикаций автора
                vakbib1,vakbib2,%
                confbib1,confbib2,%
                bib1,bib2,%
        }%
        \publications\ Основные результаты по теме диссертации изложены в \arabic{citeauthor} печатных изданиях, 
        \arabic{citeauthorvak} из которых изданы в журналах, рекомендованных ВАК, 
        \arabic{citeauthorconf} "--- в тезисах докладов.
    \end{refsection}
    \begin{refsection}[vak,papers,conf]%Блок, позволяющий отобрать из всех работ автора наиболее значимые, и только их вывести в автореферате, но считать в блоке выше общее число работ
        \printbibliography[heading=countauthorvak, env=countauthorvak, keyword=biblioauthorvak, section=2]%
        \printbibliography[heading=countauthornotvak, env=countauthornotvak, keyword=biblioauthornotvak, section=2]%
        \printbibliography[heading=countauthorconf, env=countauthorconf, keyword=biblioauthorconf, section=2]%
        \printbibliography[heading=countauthor, env=countauthor, keyword=biblioauthor, section=2]%
        \nocite{vakbib2}%vak
        \nocite{bib1}%notvak
        \nocite{confbib1}%conf
    \end{refsection}
}
При использовании пакета \verb!biblatex! для автоматического подсчёта
количества публикаций автора по теме диссертации, необходимо
их~здесь перечислить с использованием команды \verb!\nocite!.
