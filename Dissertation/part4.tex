\chapter{Мониторинг дрейфа льда и моделирование дрейфа айсбергов на основе результатов анализа SAR"--~изображений} \label{chapt4}

В данной главе описывается гидродинамическая модель дрейфа айсбергов, и приводятся примеры ее использования в оперативной работе по прогнозированию дрейфа айсбергов в Карском море, и в контексте усовершенствования ледового мониторинга в Баренцевом море с использованием данных о местоположении айсбергов, полученном на основе автоматизированного анализа SAR"--~изображений.

Мониторинг динамических характеристик льда "--- дрейфа и деформации, рассматривается на примере разрабюотанной автором информационной подсистемы потоковой обработки спутниковой SAR"--~информации.

\section{Моделирование дрейфа айсбергов как часть ледового мониторинга в Западной Арктике} \label{sect4_1}

Айсберги наблюдаются в большинстве районов морей Западной Арктики, и встреча с ними является одной из наиболее реальных и опасных угроз для судов и морских производственных объектов~\cite{Mironov_Smirnov_Iceberg_2015}. В последние годы, в связи с активными работами по освоению нефтегазовых месторождений на шельфе Баренцева и Карского морей вопрос предотвращения айсберговой угрозы становится особенно острым. Ледовые угрозы относятся к категории управляемых, поскольку существует возможность воздействовать на тяжёлые льды и айсберги с помощью ледоколов и других средств. Это осуществляется с помощью системы управления ледовой обстановкой (УЛО) или ледового менеджмента. Ледовый мониторинг, включающий наблюдения, оценку и прогноз возможных изменений ледовых условий является ключевым компонентом УЛО.
Численное гидродинамическое моделирование может оказать существенную помощь при организации системы ледового мониторинга в Западной Арктике. Гидродинамические модели являются основой прогнозирования движения и трансформации айсбергов, помогают ранжировать источники айсбергов по степени их опасности для исследуемого района. В настоящем разделе описываются возможности применения гидродинамической модели дрейфа айсбергов для оперативного обеспечения морской деятельности на шельфе и для исследовательских работ по оптимизации системы ледового мониторинга в Западной Арктике с использованием данных о местоположении айсбергов, полученных на основе автоматизированного анализа SAR"--~данных.

\subsection{Краткое описание модели дрейфа айсбергов} \label{subsect4_1_1}

Уравнение баланса сил, действующих на айсберг, дрейфующий в поверхностном слое воды, запишем в виде:

\begin{equation}
\label{eq:equation4_1}
m{\frac{\vec{\partial W_i}}{\partial t}} = \vec{F_a}+\vec{F_w}+\vec{F_{if}}+\vec{F_p}+\vec{F_C},
\end{equation}
где $\vec{W_i}$ "--- скорость дрейфа айсберга, $m$ "--- его масса, $\vec{F_a}$ "--- сила воздействия ветра, $\vec{F_w}$ "--- сила воздействия течения, $\vec{F_{if}}$ "--- сила воздействия дрейфующего льда, $\vec{F_C}$ "--- сила Кориолиса, $t$ "--- время.

Силу воздействия ветра на айсберг запишем в соответствии с классической квардратической зависимостью от скорости ветра:

\begin{equation}
\label{eq:equation4_2}
\vec{F_a} = \rho_{a}S_{a}C_{a}\left|{\vec{W_a}}\right|\left(\vec{W_a}\right),
\end{equation}
где $\vec{W_a}$ "--- скорость ветра, $\rho_{a}$ "--- плотность воздуха, $S_{a}$ "--- надводная площадь айсберга перпендикулярная направлению ветра, $C_{a}$ "---  эмпирический коэффициент.

Силу воздействия течений на айсберг выразим квардратической зависимостью от разницы скорости течения и дрейфа айсберга:

\begin{equation}
\label{eq:equation4_3}
\vec{F_w} = \rho_{w}S_{w}C_{w}\left|{\vec{W_w}-\vec{W_i}}\right|\left({\vec{W_w}-\vec{W_i}}\right),
\end{equation}
где $\vec{W_w}$ "--- средняя в слое от поверхности до нижней границы айсберга скорость течения, $\rho_{w}$ "--- плотность воды, $S_{w}$ "--- подводная площадь айсберга перпендикулярная направлению среднего течения, $C_{w}$ "---  эмпирический коэффициент.

Силу воздействия дрейфующего льда на айсберг выразим квардратической зависимостью от разницы скорости дрейфа льда и скорости дрейфа айсберга:

\begin{equation}
\label{eq:equation4_4}
\vec{F_{if}} = \rho_{i}S_{if}C_{if}\left|{\vec{W_{if}}-\vec{W_i}}\right|\left({\vec{W_{if}}-\vec{W_i}}\right),
\end{equation}
где $\vec{W_{if}}$ "--- скорость дрейфа льда, $\rho_{i}$ "--- плотность льда, $S_{if}$ "--- площадь айсберга, находящаяся в соприкосновении с дрейфующим льдом и перпендикулярно направлению его дрейфа, $C_{if}$ "---  эмпирический коэффициент.

Силу градиента давления определим через проекцию силы тяжести на горизонтальную поверхность:

\begin{equation}
\label{eq:equation4_5}
\vec{F_{p}} = gm\vec{\nabla}\zeta,
\end{equation}
где $g$ "--- ускорение свободного падения, $\zeta$ "--- превышение уровня моря над невозмущенной поверхностью.

Силу Кориолиса запишем в классическом виде:

\begin{equation}
\label{eq:equation4_6}
\vec{F_{c}} = \Omega m\vec{k}\times \vec{W_i},
\end{equation}
где $\Omega$ "--- угловая скорость вращения, $k$ "--- единичный вектор, направленный вертикально вверх.

Сформулированная задача (\labelcref{eq:equation4_1,eq:equation4_2,eq:equation4_3,eq:equation4_4,eq:equation4_5,eq:equation4_6}) позволяет средствами численного математического моделирования определить скорость, а в дальнейшем и траекторию дрейфа айсберга при известных форсингах. Однако если существует несколько доступных путей получения скорости ветра, то такие параметры, как скорость течения, скорость дрейфа льда и денивеляция уровенной поверхности моря, можно получить только путем их расчета в модели совместной циркуляции вод и льдов. В данном исследовании в качестве такой модели будем использовать AARI"--~IOCM~\cite{Kulakov2012}. AARI"--~IOCM представляет собой результат объединения трех моделей: трехмерной бароклинной модели циркуляции вод, модели дрейфа ледяного покрова и термодинамической модели морского льда. Модель адаптирована к акватории Северного Ледовитого океана (СЛО) и прилежащей акватории Атлантического океана и имеет пространственное разрешение 13,8 км. Размер сеточной области 440 на 395 точек. По вертикали разрешение переменное, расчет производится на 33 горизонтах. Для описания донной топографии и конфигурации береговой черты использован архив GEBCO. В качестве граничных условий используются среднемесячные среднемноголетние значения расходов 17 основных рек, впадающих в СЛО. Температура и соленость воды из World Ocean Atlas ( WOA05) для летнего или зимнего периодов взяты в качестве начальных условий. В качестве внешнего форсинга используются данные об атмосферном давлении на уровне моря и температуре воздуха на высоте 2 м из архива NCEP/NCAR для диагностических расчетов или прогностические данные Европейского центра среднесрочных прогнозов погоды (ECMWF), представленные на сетке $2,5^\circ \times 2,5^\circ$.

Модель была верифицирована по данным радиомаяков ARGOS установленных на айсберги в экспедиции <<Кара"--~зима-2013>>. Результаты расчетов показали, что модель удовлетворительно воспроизводит дрейф айсбергов (рис.~\ref{img:ibg_validation_01}).

\begin{figure}[ht] 
	\centering
	\includegraphics [scale=0.08] {ibg_validation_01}
	\caption{Сопоставление наблюденных (жирная линия) и рассчитанных по модели (тонкая линия) траекторий айсбергов в 2013 г.}
	\label{img:ibg_validation_01}
\end{figure}

\subsection{Оперативное обеспечение разведочного бурения в Карском море.}\label{subsect4_1_2}
На основе описанной выше модели была создана технология прогнозирования айсбергов \cite{Mironov2015}, позволяющая в автоматическом режиме производить прогнозы их дрейфа на срок до 5 суток. Модель была существенно модифицирована. Для уменьшения пространственного шага модели AARI"--~IOCM использовалась процедура телескопирования, при которой на всей акватории СЛО используется обычный большой шаг (в нашем случае 13,8 км), а на акватории, для которой проводится прогнозирование, в несколько раз меньше. Для акватории Карского моря шаг модели составляет 4,6 км.

Для лучшего воспроизведения атмосферного форсинга в модельный комплекс включена атмосферная региональная модель WRF. В настоящий момент версия модели Polar WRF~\cite{bromwich2009development} адаптирована для района, покрывающего Юго"--~Западную часть Карского моря и Печорское море с горизонтальным разрешением 4 км.
Разработаны скрипты постобработки на языке Python формирующие линейные shape файлы, которые являются конечным продуктом технологической цепочки. Эти файлы позволяют потребителям информации оперативно просматривать результаты расчетов, а также контролировать их качество.

Технология численного гидродинамического прогноза дрейфа айсбергов была успешно апробирована при оперативном обеспечении геологоразведочных работ на шельфе Карского моря. В течение августа"--~октября 2014 г. осуществлялось разведочное бурение на геологической структуре <<Университетская"--~1>> на лицензионном участке Восточно"--~Приновоземельский"--~1 в Карском море. Бурение производилось морской полупогружной плавучей буровой установкой (ППБУ) <<West Alpha>>, не имеющей ледового класса, что обусловило повышенные требования к ее ледовому и гидрометеорологическому обеспечению. Специалисты Арктического и Антарктического Научно"--~Исследовательского Института (<<ААНИИ>>) осуществляли полный комплекс работ по мониторингу окружающей среды и гидрометеорологическому обеспечению руководства морской операцией и капитана ППБУ. Главной задачей ледового мониторинга было обнаружение опасных ледяных образований (ледяные поля, несяки, айсберги), предсказание их дрейфа и возможного их столкновения с ППБУ. Также принималась во внимание возможность выноса с выводных ледников архипелага Новой Земли айсбергов и их обломков в район Университетской структуры~\cite{Mironov_Smirnov_Iceberg_2015}.

Прогноз дрейфа айсбергов заблаговременностью 24"--~72 часов готовился практически ежедневно и передавался, в основном, в формате шейп"--~файлов в Береговой Операционный Центр (SOC "--- Shore Operational Center), который находился в Москве. В SOC вся информация анализировалась и передавалась на ППБУ и вспомогательные суда. Пример представления прогноза дрейфа айсбергов для потребителя представлен на рис.~\ref{img:ibg_frc}.

\begin{figure}[ht] 
	\centering
	\includegraphics [scale=0.57] {ibg_frc}
	\caption{Пример прогноз дрейфа айсбергов на 9.09.2014, предоставленный в рамках ледового и гидрометеорологического обеспечения буровых работ в Карском море.}
	\label{img:ibg_frc}
\end{figure}

Оправдываемость и эффективность ледовых и гидрометеорологических прогнозов отвечала всем критериям действующих нормативных документов.

Гидрометеорологическое и ледовое обслуживания разведочного бурения в Карском море в 2014 г. обеспечило безопасность и эффективность работ, что способствовало открытию нового нефтяного месторождения <<Победа>>.  

\subsection{Реконструкция дрейфа айсбергов в район ШГКМ в 2003 г.}\label{subsect4_1_3}
Для добычных комплексов на Штокмановском газоконденсатном месторождении (ШГКМ) в Баренцевом море наиболее опасными ледяными образованиями будут дрейфующие айсберги. По данным наблюдений за периоды 1928"--~1992 и 2002"--~2005 гг. на акватории, прилегающей к ШГКМ, было зафиксировано 220 айсбергов и кусков айсбергов~\cite{zubakin2007results}. Эти фиксации были отмечены в 1967, 1968, 1971, 1975, 1981, 1986, 1987, 1989, 1991 и 2003 г. 

Источниками айсбергов, распространяющихся на акватории Баренцева моря, являются арктические архипелаги Шпицберген, Земля Франца"--~Иосифа, Новая Земля (о. Северный) и некоторые арктические острова (о. Ушакова и о. Виктория) и, даже Северная Земля~\cite{Koryakin1988,sandford1955tabular}. Важной составляющей ледового мониторинга является задача своевременного обнаружения айсбергов, потенциально опасных для добывающей платформы. Эта задача может быть успешно решена только в том случае, когда все источники айсбергов будут ранжированы по степени их угрозы.

За прошедшие годы было предпринято несколько попыток использовать математическое моделирование для оценки айсберговой опасности. Так, в~\cite{johannessen1999simulation} приводятся результаты модельных расчетов дрейфа айсбергов от южного побережья архипелага Земля Франца"--~Иосифа. Авторы рассчитали траектории 3000 айсбергов стартовавших в июле, августе и сентябре за период от 1987 по 1996 гг. Согласно их результатам, ни один айсберг не достиг района ШГКМ. Большинство айсбергов дрейфовало на запад к архипелагу Шпицберген, некоторое количество айсбергов уходило в Центральный Арктический бассейн и только несколько дрейфовало на восток к северной оконечности Новой Земли.

Наблюдения за айсбергами выполненные в рамках экспедиции <<ШТОКМАН"--~ЗИМА"--~2003>>~\cite{Naumov2003}, когда непосредственно через площадь ШГКМ продрейфовало 104 айсберга и их обломков, свидетельствовали об обратном. По некоторым косвенным признакам айсберги, обнаруженные в районе ШГКМ, происходили от ледников архипелага Земля Франца"--~Иосифа. 

В работе~\cite{Buzin2008} была предпринята попытка использовать моделирование для подтверждения гипотезы о том, что айсберги, обнаруженные в 2003 г. в районе ШГКМ, поступили туда от архипелага Земля Франца"--~Иосифа. Для этого была выполнена серия расчетов на гидродинамической модели~\cite{polyakov1998coupled} с подключенным к ней блоком перемещения айсбергов~\cite{Dmitriev1995}.


\begin{table} [htbp]
  \centering
  \changecaptionwidth\captionwidth{15cm}
  \caption{Название таблицы}\label{Ts0Sib}%
  \begin{tabular}{| p{3cm} || p{3cm} | p{3cm} | p{4cm}l |}
  \hline
  \hline
  Месяц   & \centering $T_{min}$, К & \centering $T_{max}$, К &\centering  $(T_{max} - T_{min})$, К & \\
  \hline
  Декабрь &\centering  253.575   &\centering  257.778    &\centering      4.203  &   \\
  Январь  &\centering  262.431   &\centering  263.214    &\centering      0.783  &   \\
  Февраль &\centering  261.184   &\centering  260.381    &\centering     $-$0.803  &   \\
  \hline
  \hline
  \end{tabular}
\end{table}

\begin{table} [htbp]% Пример записи таблицы с номером, но без отображаемого наименования
	\centering
	\parbox{9cm}{% чтобы лучше смотрелось, подбирается самостоятельно
        \captiondelim{}% должен стоять до самого пустого caption
        \caption{}%
        \label{tbl:test1}%
        \begin{SingleSpace}
    	\begin{tabular}{ | c | c | c | c |}
    	\hline
    	Оконная функция	& ${2N}$ & ${4N}$	& ${8N}$	\\ \hline
    	Прямоугольное 	& 8.72 	 & 8.77		& 8.77		\\ \hline
    	Ханна		& 7.96 	 & 7.93		& 7.93		\\ \hline
    	Хэмминга	& 8.72 	 & 8.77		& 8.77		\\ \hline
    	Блэкмана	& 8.72 	 & 8.77		& 8.77		\\ \hline
    	\end{tabular}%
    	\end{SingleSpace}
	}
\end{table}

Таблица \ref{tbl:test2} "--- пример таблицы, оформленной в~классическом книжном варианте или~очень близко к~нему. \mbox{ГОСТу} по~сути не~противоречит. Можно ещё~улучшить представление, с~помощью пакета \verb|siunitx| или~подобного.

\begin{table} [htbp]%
    \centering
	\caption{Наименование таблицы, очень длинное наименование таблицы, чтобы посмотреть как оно будет располагаться на~нескольких строках и~переноситься}%
	\label{tbl:test2}% label всегда желательно идти после caption
    \renewcommand{\arraystretch}{1.5}%% Увеличение расстояния между рядами, для улучшения восприятия.
    \begin{SingleSpace}
	\begin{tabular}{@{}@{\extracolsep{20pt}}llll@{}} %Вертикальные полосы не используются принципиально, как и лишние горизонтальные (допускается по ГОСТ 2.105 пункт 4.4.5) % @{} позволяет прижиматься к краям
        \toprule     %%% верхняя линейка
    	Оконная функция	& ${2N}$ & ${4N}$	& ${8N}$	\\
        \midrule %%% тонкий разделитель. Отделяет названия столбцов. Обязателен по ГОСТ 2.105 пункт 4.4.5 
    	Прямоугольное 	& 8.72 	 & 8.77		& 8.77		\\
    	Ханна		& 7.96 	 & 7.93		& 7.93		\\
    	Хэмминга	& 8.72 	 & 8.77		& 8.77		\\
    	Блэкмана	& 8.72 	 & 8.77		& 8.77		\\
        \bottomrule %%% нижняя линейка
	\end{tabular}%
   	\end{SingleSpace}
\end{table}

\section{Таблица с многострочными ячейками и примечанием}

Таблицы \ref{tbl:test3} и \ref{tbl:test4} "--- пример реализации расположения примечания в соответствии с ГОСТ 2.105. Каждый вариант со своими достоинствами и недостатками. Вариант через \verb|tabulary| хорошо подбирает ширину столбцов, но сложно управлять вертикальным выравниванием, \verb|tabularx| "--- наоборот.
\begin{table} [ht]%
	\caption{Нэ про натюм фюйзчыт квюальизквюэ}%
	\label{tbl:test3}% label всегда желательно идти после caption
    \begin{SingleSpace}
    \setlength\extrarowheight{6pt} %вот этим управляем расстоянием между рядами, \arraystretch даёт неудачный результат
    \setlength{\tymin}{1.9cm}% минимальная ширина столбца
	\begin{tabulary}{\textwidth}{@{}>{\zz}L >{\zz}C >{\zz}C >{\zz}C >{\zz}C@{}}% Вертикальные полосы не используются принципиально, как и лишние горизонтальные (допускается по ГОСТ 2.105 пункт 4.4.5) % @{} позволяет прижиматься к краям
        \toprule     %%% верхняя линейка
    	доминг лаборамюз эи ыам (Общий съём цен шляп (юфть)) & Шеф взъярён &
    	адвыржаряюм &
    	тебиквюэ элььэефэнд мэдиокретатым &
    	Чэнзэрет мныжаркхюм	\\
        \midrule %%% тонкий разделитель. Отделяет названия столбцов. Обязателен по ГОСТ 2.105 пункт 4.4.5 
         Эй, жлоб! Где туз? Прячь юных съёмщиц в~шкаф Плюш изъят. Бьём чуждый цен хвощ! &
        ${\approx}$ &
        ${\approx}$ &
        ${\approx}$ &
        $ + $ \\
        Эх, чужак! Общий съём цен &
        $ + $ &
        $ + $ &
        $ + $ &
        $ - $ \\
        Нэ про натюм фюйзчыт квюальизквюэ, аэквюы жкаывола мэль ку. Ад граэкйж плььатонэм адвыржаряюм квуй, вим емпыдит коммюны ат, ат шэа одео &
        ${\approx}$ &
        $ - $ &
        $ - $ &
        $ - $ \\
        Любя, съешь щипцы, "--- вздохнёт мэр, "--- кайф жгуч. &
        $ - $ &
        $ + $ &
        $ + $ &
        ${\approx}$ \\
        Нэ про натюм фюйзчыт квюальизквюэ, аэквюы жкаывола мэль ку. Ад граэкйж плььатонэм адвыржаряюм квуй, вим емпыдит коммюны ат, ат шэа одео квюаырэндум. Вёртюты ажжынтиор эффикеэнди эож нэ. &
        $ + $ &
        $ - $ &
        ${\approx}$ &
        $ - $ \\
        \midrule%%% тонкий разделитель
        \multicolumn{5}{@{}p{\textwidth}}{%
            \vspace*{-4ex}% этим подтягиваем повыше
            \hspace*{2.5em}% абзацный отступ - требование ГОСТ 2.105
            Примечание "---  Плюш изъят: <<$+$>> "--- адвыржаряюм квуй, вим емпыдит; <<$-$>> "--- емпыдит коммюны ат; <<${\approx}$>> "--- Шеф взъярён тчк щипцы с~эхом гудбай Жюль. Эй, жлоб! Где туз? Прячь юных съёмщиц в~шкаф. Экс-граф?
        }
        \\
        \bottomrule %%% нижняя линейка
	\end{tabulary}%
    \end{SingleSpace}
\end{table}

Из-за того, что таблица \ref{tbl:test3} не помещается на той же странице (при компилировании pdflatex), всё её содержимое переносится на следующую, ближайшую, а~этот текст идёт перед ней.
\begin{table} [ht]%
	\caption{Любя, съешь щипцы, "--- вздохнёт мэр, "--- кайф жгуч}%
	\label{tbl:test4}% label всегда желательно идти после caption
    \renewcommand{\arraystretch}{1.6}%% Увеличение расстояния между рядами, для улучшения восприятия.
	\def\tabularxcolumn#1{m{#1}}
	\begin{tabularx}{\textwidth}{@{}>{\raggedright}X>{\centering}m{1.9cm} >{\centering}m{1.9cm} >{\centering}m{1.9cm} >{\centering\arraybackslash}m{1.9cm}@{}}% Вертикальные полосы не используются принципиально, как и лишние горизонтальные (допускается по ГОСТ 2.105 пункт 4.4.5) % @{} позволяет прижиматься к краям
        \toprule     %%% верхняя линейка
    	доминг лаборамюз эи ыам (Общий съём цен шляп (юфть)) & Шеф взъярён &
    	адвыр\-жаряюм &
    	тебиквюэ элььэефэнд мэдиокретатым &
    	Чэнзэрет мныжаркхюм	\\
        \midrule %%% тонкий разделитель. Отделяет названия столбцов. Обязателен по ГОСТ 2.105 пункт 4.4.5 
         Эй, жлоб! Где туз? Прячь юных съёмщиц в~шкаф Плюш изъят. Бьём чуждый цен хвощ! &
        ${\approx}$ &
        ${\approx}$ &
        ${\approx}$ &
        $ + $ \\
        Эх, чужак! Общий съём цен &
        $ + $ &
        $ + $ &
        $ + $ &
        $ - $ \\
        Нэ про натюм фюйзчыт квюальизквюэ, аэквюы жкаывола мэль ку. Ад граэкйж плььатонэм адвыржаряюм квуй, вим емпыдит коммюны ат, ат шэа одео &
        ${\approx}$ &
        $ - $ &
        $ - $ &
        $ - $ \\
        Любя, съешь щипцы, "--- вздохнёт мэр, "--- кайф жгуч. &
        $ - $ &
        $ + $ &
        $ + $ &
        ${\approx}$ \\
        Нэ про натюм фюйзчыт квюальизквюэ, аэквюы жкаывола мэль ку. Ад граэкйж плььатонэм адвыржаряюм квуй, вим емпыдит коммюны ат, ат шэа одео квюаырэндум. Вёртюты ажжынтиор эффикеэнди эож нэ. &
        $ + $ &
        $ - $ &
        ${\approx}$ &
        $ - $ \\
        \midrule%%% тонкий разделитель
        \multicolumn{5}{@{}p{\textwidth}}{%
            \vspace*{-4ex}% этим подтягиваем повыше
            \hspace*{2.5em}% абзацный отступ - требование ГОСТ 2.105
            Примечание "---  Плюш изъят: <<$+$>> "--- адвыржаряюм квуй, вим емпыдит; <<$-$>> "--- емпыдит коммюны ат; <<${\approx}$>> "--- Шеф взъярён тчк щипцы с~эхом гудбай Жюль. Эй, жлоб! Где туз? Прячь юных съёмщиц в~шкаф. Экс-граф?
        }
        \\
        \bottomrule %%% нижняя линейка
	\end{tabularx}%
\end{table}

%\newpage
%============================================================================================================================

\section{Параграф - два} \label{sect4_2}

Некоторый текст.

%\newpage
%============================================================================================================================

\section{Параграф с подпараграфами} \label{sect4_3}

\subsection{Подпараграф - один} \label{subsect4_4_1}

Некоторый текст.

\subsection{Подпараграф - два} \label{subsect4_4_2}

Некоторый текст.

\clearpage